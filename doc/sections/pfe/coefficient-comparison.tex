\section{Coefficient Comparison}

Given a rational  function  $T(s)$,  the partial fraction expansion problem is
finding suitable coefficients ${A_1,A_2,\ldots,A_n}$ to  satisfy the equation:

\begin{equation}
    T(s) = \frac{b_0 + b_1s + b_2s^2 + \ldots + b_ns^n}{(s-p_1)(s-p_2)\ldots(s-p_n)} = \frac{A_1}{s-p_1} + \frac{A_2}{s-p_2} + \ldots + \frac{A_n}{s-p_n}
\end{equation}

Given that we know the numerator coefficients $b_n$  and  we  know  the  poles
$p_n$.

One way to achieve this is to bring everything to a common denominator, factor
in $A_n$ and compare the coefficients $b_n$ with the coefficients $A_n,p_n$ to
set up a linear system of equations.


\subsection{Second Order Example}

To help us understand this,  we  begin  with  an  example  with a second order
transfer function:

\begin{equation}
    T(s) = \frac{b_0 + b_1s}{(s-p_1)(s-p_2)} = \frac{A_1}{s-p_1} + \frac{A_2}{s-p_2}
\end{equation}

First, bring everything to a common denominator:

\begin{equation}
    \frac{A_1}{(s-p_1)}\cdot\frac{s-p_2}{s-p_2} + \frac{A_2}{(s-p_2)}\cdot\frac{s-p_1}{s-p_1} = \frac{A_1s - A_1p_2 + A_2s - A_2p_1}{(s-p_1)(s-p_2)}
\end{equation}

Then factor the terms for $s$:

\begin{equation}
    \frac{A_1s - A_1p_2 + A_2s - A_2p_1}{(s-p_1)(s-p_2)} = \frac{(-A_2p_1 -A_1p_2) + (A_1+A_2)s}{(s-p_1)(s-p_2)}
\end{equation}

Now we can perform the actual coefficient comparison. We note that the problem
has been reduced to the following equation:

\begin{equation}
    b_0 + b_1s = (-A_2p_1-A_1p_2) + (A_1+A_2)s
\end{equation}

which can be reformulated into a linear system of equations:

\begin{align}
    s^1: b_1 &= A_1 + A_2 \\
    s^0: b_0 &= -A_1p_2 -A_2p_1
\end{align}

or, in matrix form:

\begin{equation}
    \begin{bmatrix}
        b_1 \\
        b_0 \\
    \end{bmatrix} =
    \begin{bmatrix}
        1    & 1    \\
        -p_2 & -p_1 \\
    \end{bmatrix}
    \begin{bmatrix}
        A_1 \\
        A_2 \\
    \end{bmatrix}
\end{equation}


\subsection{Third Order Example}

Let's  try  the same thing but with a third order transfer  function.  Due  to
space limitations, the steps will not be presented,  only the resulting linear
system of equations, but they are the exact  same  as  with  the  second order
example.

\begin{equation}
    T(s) = \frac{b_0 + b_1s + b_2s^2}{(s-p_1)(s-p_2)(s-p_3)} = \frac{A_1}{s-p_1} + \frac{A_2}{s-p_2} + \frac{A_3}{s-p_3}
\end{equation}

Bring  everything  to  a  common  denominator,  factor  out  $s$  and  compare
coefficients.  This   yields   the   following  linear  system  of  equations:

\begin{align}
    s^2: b_2 &= A_1+A_2+A_3 \\
    s^1: b_1 &= A_1(-p_2-p_3) + A_2(-p_1-p_3) + A_3(-p_1-p_2) \\
    s^0: b_0 &= A_1p_2p_3 + A_2p_1p_3 + A_3p_1p_2
\end{align}

or, in matrix form:

\begin{equation}
    \begin{bmatrix}
        b_2 \\
        b_1 \\
        b_0 \\
    \end{bmatrix} =
    \begin{bmatrix}
        1        & 1        & 1 \\
        -p_2-p_3 & -p_1-p_3 & -p_1-p_2 \\
        p_2p_3   & p_1p_3   & p_1p_2 \\
    \end{bmatrix}
    \begin{bmatrix}
        A_1 \\
        A_2 \\
        A_3 \\
    \end{bmatrix}
\end{equation}


\subsection{Fourth Order Example}

It  is  still  not  quite  clear yet what the pattern is, so let's do one more
example with a fourth order transfer function:

\begin{align}
    T(s) &= \frac{b_0 + b_1s + b_2s^2 + b_3s^3}{(s-p_1)(s-p_2)(s-p_3)(s-p_4)} \\
         &= \frac{A_1}{s-p_1} + \frac{A_2}{s-p_2} + \frac{A_3}{s-p_3} + \frac{A_4}{s-p_4}
\end{align}

This produces the following system of equations:

\begin{align*}
    s^3: b_3 &= A_1 + A_2 + A_3 + A_4 \\
    s^2: b_2 &= A_1(-p_2-p_3-p_4) + A_2(-p_1-p_3-p_4) \\
             &= + A_3(-p_1-p_2-p_4) + A_4(-p_1-p_2-p_3) \\
    s^1: b_1 &= A_1(p_2p_3+p_2p_4+p_3p_4) + A_2(p_1p_3+p_1p_4+p_3p_4) \\
             &= + A_3(p_1p_2+p_1p_4+p_2p_4) + A_4(p_1p_2+p_1p_3+p_2p_3) \\
    s^0: b_0 &= A_1(-p_2p_3p_4) + A_2(-p_1p_3p_4) + A_3(-p_1p_2p_4) + A_4(-p_1p_2p_3)
\end{align*}

Which is now more readable in matrix form:

\begin{equation}
    \begin{bmatrix}
        b_3 \\
        b_2 \\
        b_1 \\
        b_0 \\
    \end{bmatrix} =
    \begin{bmatrix}
        1                    & 1                    & 1                    & 1 \\
        -p_2-p_3-p_4         & -p_1-p_3-p_4         & -p_1-p_2-p_4         & -p_1-p_2-p_3 \\
        p_2p_3+p_2p_4+p_3p_4 & p_1p_3+p_1p_4+p_3p_4 & p_1p_2+p_1p_4+p_2p_4 & p_1p_2+p_1p_3+p_2p_3 \\
        -p_2p_3p_4           & -p_1p_3p_4           & -p_1p_2p_4           & -p_2p_3p_4 \\
    \end{bmatrix}
    \begin{bmatrix}
        A_1 \\
        A_2 \\
        A_3 \\
        A_4 \\
    \end{bmatrix}
\end{equation}


\subsection{Multi-Roots}

\begin{equation}
    \frac{b_0 + b_1s + b_2s^2 + b_3s^3}{(s-p_1)^4} = \frac{A_1}{(s-p_1)^4} + \frac{A_2}{(s-p_1)^3} + \
